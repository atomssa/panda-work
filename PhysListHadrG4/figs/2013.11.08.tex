\documentclass{beamer}
%\usepackage{beamerthemeshadow}
\usetheme{Madrid}

%\usepackage[utf8]{inputenc}
%\usepackage{xspace}
%\usepackage[T1]{fontenc}

\usepackage{amsmath}
\usepackage{amssymb}


\begin{document}

\renewcommand{\inserttotalframenumber}{16}

\title[Physics List Comparisons]{Physics List Comparisons}
\author[Ermias ATOMSSA]{Ermias ATOMSSA}
\date{\today}
\institute[IPNO]{Institut de Physique Nucleaire d'Orsay}

\newcommand{\dm}{\mathrm{d}m}
\newcommand{\dpt}{\mathrm{d}p_T} 
\newcommand{\pt}{p_T}
\newcommand{\pta}{p^a_T}
\newcommand{\ptb}{p^b_T}
\newcommand{\dph}{\Delta\phi}
\newcommand{\jaa}{J_{AA}}
\newcommand{\iaa}{I_{AA}}
\newcommand{\raa}{R_{AA}}
\newcommand{\ncoll}{N_{coll}}

\setlength{\belowdisplayskip}{0pt} \setlength{\belowdisplayshortskip}{0pt}
\setlength{\abovedisplayskip}{0pt} \setlength{\abovedisplayshortskip}{0pt}


\frame{\titlepage}


\frame{
  \frametitle{Objective of study}
  \begin{itemize}
    \scriptsize
  \item Understand the implication of using various available physics lists for Geant 4
    \begin{itemize}
      \scriptsize
    \item Previous PiD algorithm work done using Geant 3
    \item Feasibility of physics measurements $\bar{p}p\rightarrow\pi^0e^+e^-$ critically dependent on PiD
    \item Tail of hadronic response plays dominant role in hadron rejection 
    \end{itemize}
    \vspace{0.3cm}
  \item How does Geant 4 physics lists compare to the default option of Geant 3 (used so far)? 
    \vspace{0.3cm}
  \item Strategy: Select Geant 4 physics list with largest tail to be conservative? 
    \vspace{0.3cm}
  \item Geant 4 also offers other options with more or less cpu cost. What to use? 
    \begin{itemize}
      \scriptsize
    \item ``Optical physics'' simulation (very high cpu/storage cost)
    \item ``High precision neutron data'' (very high cpu)
    \item Most accurate parametrization of EM physics vs. tuned (for cpu) less accurate one 
    \end{itemize}
    \vspace{0.3cm}
  \item PandaROOT setup
    \begin{itemize}
      \scriptsize
    \item Full PANDA simulation vs. EMCal only simulation
    \end{itemize}
  \end{itemize}

}

\frame{
  \frametitle{PandaROOT setup}
  \begin{itemize}
    \scriptsize
  \item Events
    \begin{itemize}
      \scriptsize
    \item 50k $\pi^+$ and 50k $\pi^-$ for each physics list
    \item Uniform in $\phi\in$ (0, 360$^\circ$) and $\theta\in$ (85$^\circ$, 95$^\circ$)
    \item Acceptance cut to exclude $\phi\in$ (-100$^\circ$,90$^\circ$) and  $\phi\in$ (90$^\circ$,100$^\circ$)
    \item Each setup at 5 different momenta (in GeV/c: 0.5, 0.8, 1.0, 1.5, 3.0 and 5.0)
    \item All tracks start from $(v_x,v_y,v_z)=0.0$
    \end{itemize}
    \vspace{0.1cm}
  \end{itemize}
  \only<2>{
    \begin{figure}
      \includegraphics[width=0.65\columnwidth]{tc.eps}
    \end{figure}
  }
  \only<1>{
    \begin{itemize}
    \item Detector setup for transport stage
      \begin{itemize}
        \scriptsize
      \item More details on Geant physics lists next slide
      \item EMCal only setup used for most comparisons
      \item For sanity check, full panda setup compared to EMCal only in a few setups
      \end{itemize}
      \vspace{0.1cm}
    \item Reconstruction of the EMCal
      \begin{itemize}
        \scriptsize
      \item Full EMCal reconstruction used with the following modules: \\
        {\tiny PndEmcHitProducer, PndEmcHitsToWaveform, PndEmcWaveformToDigi, \\
          PndEmcMakeCluster, PndEmcHdrFiller, PndEmcMakeBump, PndEmcMakeRecoHit.}
      \item Bug/Feature? Cluster, Bump and RecoHit containers have
        exactly the same multiplicity and same energy for all
        objects.
      \end{itemize}
      \vspace{0.1cm}
    \item Plotted quantity: $E_{reco}/E_{true}$
      \begin{itemize}
        \scriptsize
      \item $E_{true}$: Energy of simulated pion track
      \item $E_{reco}$: Energy of EmcRecoHit object with closest $\theta_{reco}$ to $\theta_{origin}$ of simulated pion
      \end{itemize}
    \end{itemize}
  }
}


%\frame{
%  \frametitle{PandaROOT setup}
%  \begin{itemize}
%    \scriptsize
%  \item Events
%    \begin{itemize}
%      \scriptsize
%    \item 50k $\pi^+$ and 50k $\pi^-$ for each physics list
%    \item Uniform in $\phi\in$ (0, 360$^\circ$) and $\theta\in$ (85$^\circ$, 95$^\circ$)
%    \item Acceptance cut to exclude $\phi\in$ (-100$^\circ$,90$^\circ$) and  $\phi\in$ (90$^\circ$,100$^\circ$)
%    \item Each setup at 5 different momenta (in GeV/c: 0.5, 0.8, 1.0, 1.5, 3.0 and 5.0)
%    \item All tracks start from $(v_x,v_y,v_z)=0.0$
%    \end{itemize}
%    \vspace{0.1cm}
%  \item Detector setup for transport stage
%    \begin{itemize}
%      \scriptsize
%    \item More details on Geant physics lists next slide
%    \item EMCal only setup used for most comparisons
%    \item For sanity check, full panda setup compared to EMCal only in a few setups
%    \end{itemize}
%    \vspace{0.1cm}
%  \item Reconstruction of the EMCal
%    \begin{itemize}
%      \scriptsize
%    \item Full EMCal reconstruction used with the following modules: \\
%      {\tiny PndEmcHitProducer, PndEmcHitsToWaveform, PndEmcWaveformToDigi, \\
%        PndEmcMakeCluster, PndEmcHdrFiller, PndEmcMakeBump, PndEmcMakeRecoHit.}
%    \item Bug/Feature? Cluster, Bump and RecoHit containers have
%      exactly the same multiplicity and same energy for all
%      objects.
%    \end{itemize}
%    \vspace{0.1cm}
%  \item Plotted quantity: $E_{reco}/E_{true}$
%    \begin{itemize}
%      \scriptsize
%    \item $E_{true}$: Energy of simulated pion track
%    \item $E_{reco}$: Energy of EmcRecoHit object with closest $\theta_{reco}$ to $\theta_{origin}$ of simulated pion
%    \end{itemize}
%  \end{itemize}
%}

\frame{
  \frametitle{Hadronic Physics Lists}
  Geant 3
  \begin{itemize}
    \scriptsize
  \item Limited number of choices for hadronic physics lists
  \item Default: HADR=3 (GEANH+NUCRIN)
  \item Other choices: HADR=4 ({\bf FLUKA}), HADR=5 ({\bf MICAP}), HADR=6 ({\bf GCALOR})
  \end{itemize}
  Geant 4
  \begin{itemize}
    \scriptsize
  \item Options depending on hadronic interaction and cascade (nuclear de-excitation) model
  \item Two ``large'' classes: Quark Gluon String (QGS) and Fritiof (FTF)
  \item Variations based on low energy hadronic interaction and nuclear de-excitation (cascade)
    \begin{itemize}
      \scriptsize
    \item {\bf QGSP\_BERT} and {\bf FTFP\_BERT} (Bertini for LE interaction)\\
      $\rightarrow$ P=``Precompound model'': HE parametrisation for nuclear dexcitation
    \item {\bf QGSP\_BIC} (Binary cascade for LE interaction) \\
      $\rightarrow$ P same as above, no FTFP\_BIC list available
    \item {\bf QGS\_BIC} and {\bf FTF\_BIC} \\
      $\rightarrow$ (Binary cascade used for nuclear de-excitation for the high energy model)
    \item High precision neutron data versions: \\
      {\bf QGSP\_BERT\_HP}, {\bf QGSP\_BIC\_HP} {\bf FTFP\_BERT\_HP}
    \end{itemize}
  \item Soon to be deleted in Geant 4.10
    \begin{itemize}
      \scriptsize
    \item {\bf LHEP}: both high and low energy interactions use parametrized models
    \item Chiral Invariant Phase Space (CHIPS) model for all nuclear de-excitations {\bf QGSC\_BERT}, {\bf QGSC}
    \end{itemize}
  \end{itemize}
}

\frame{

  \frametitle{Hadronic Physics List Table}

  %QGSP\_BERT
  %PionPlusInelastic  Models:
  %QGSP: Emin(GeV)=   12  Emax(GeV)= 100000
  %G4LEPionPlusInelastic: Emin(GeV)=  9.5  Emax(GeV)= 25
  %BertiniCascade: Emin(GeV)=    0  Emax(GeV)= 9.9
  %
  %QGSP\_BIC
  %PionPlusInelastic  Models:
  %QGSP: Emin(GeV)=   12  Emax(GeV)= 100000
  %G4LEPionPlusInelastic: Emin(GeV)=    0  Emax(GeV)= 25
  %
  % FTFP\_BERT
  %PionPlusInelastic  Models:
  %FTFP: Emin(GeV)=    4  Emax(GeV)= 100000
  %BertiniCascade: Emin(GeV)=    0  Emax(GeV)= 5
  %
  % QGS\_BIC
  %PionPlusInelastic  Models:
  %QGSB: Emin(GeV)=   12  Emax(GeV)= 100000
  %Binary Cascade: Emin(GeV)=    0  Emax(GeV)= 1.3
  %G4LEPionPlusInelastic: Emin(GeV)=  1.2  Emax(GeV)= 25
  % FTF\_BIC
  %PionPlusInelastic  Models:
  %FTFB: Emin(GeV)=    4  Emax(GeV)= 100000
  %Binary Cascade: Emin(GeV)=    0  Emax(GeV)= 5

  
  \begin{center}
    \scriptsize
    \begin{tabular}{|c|c|c|c|c|c|c|c|}
      \hline
        & \multicolumn{3}{|c|}{Low Energy}  & \multicolumn{3}{|c|}{High Energy} & \\
      \hline
       Phys. List & h-N & de-ex. & R($\pi^{\pm}$) &  h-N  &   de-ex. & R($\pi^\pm$)  & C   \\
      \hline                                                                     
      \hline                                                                     
      QGSP\_BERT           &  Bert. & Bert. & 0 - 9.9 & QGS  & Prec. &  12 - $\infty$ & LEP: 9.5 - 25 \\
      \hline
      QGSP\_BIC            &  LEP  & LEP  & 0 - 9.9 & QGS  & Prec. & 12 - $\infty$ & G4 Bug? \\
      \hline
      QGS\_BIC             &  Bin. & Bin. & 0 - 1.3 & QGS  & Bin.  & 12 - $\infty$ & LEP: 1.2 - 25 \\
      \hline
      FTFP\_BERT           &  Bert. & Bert. & 0 - 5 & FTF  & Prec. & 4 - $\infty$ & \\
      \hline
      FTF\_BIC             &  Bin. & Bin. & 0 - 5 & FTF & Bic & 4 - $\infty$ & \\
      \hline
    \end{tabular}
  \end{center}

  \hfill
  {
    \tiny
    
    h-N: Hadron-Nucleus interaction\\
    de-ex: Nuclear de-excitation\\
    R: Range\\
    C: Comments (patching model when gap between LE \& HE)\\
    LEP: Low Energy Parametrization\\
  }
}



\frame{
  \frametitle{Full PANDA simulation vs. EMCal only}
  \vskip -.4cm
  \begin{figure}
    \includegraphics[width=0.75\columnwidth]{fvse.eps}
  \end{figure}
  \begin{itemize}
    \scriptsize
  \item Slight difference at low $E_{reco}/E_{true}$ but the high end tail looks very similar
  \item Large gain in CPU usage with EMCal only simulation
  \item For purpose of comparison, will use EMCal only simulation consistently
  \end{itemize}

}


\frame{
  \frametitle{}
  
  \begin{block}{}
    \begin{center}
      \Huge
      Momentum dependence
    \end{center}
  \end{block}

}



\frame{
  \frametitle{$\pi^+$ and $\pi^-$ G3 HADR=3 (GEISHA,NUCRIN) }
  \vskip -0.3cm
  \begin{figure}
    \includegraphics[width=0.85\columnwidth]{pvsm_G3_HADR3_NUCRIN.eps}
  \end{figure}
}

\frame{
  \frametitle{$\pi^+$ and $\pi^-$ G4 QGSP, Binary cascade, param. EM processes (QGSP\_BIC\_EMV) }
  \vskip -0.3cm
  \begin{figure}
    \includegraphics[width=0.85\columnwidth]{pvsm_QGSP_BIC_EMV.eps}
  \end{figure}
}

\frame{
  \frametitle{$\pi^+$ and $\pi^-$ G4 QGSP, Bertini cascade, param. EM processes (QGSP\_BERT\_EMV) }
  \vskip -0.3cm
  \begin{figure}
    \includegraphics[width=0.85\columnwidth]{pvsm_QGSP_BERT_EMV.eps}
  \end{figure}
}



\frame{
  \frametitle{}
  
  \begin{block}{}
    \begin{center}
      \huge
      Performance tuned EM parameters and optical physics options
    \end{center}
  \end{block}

}

\frame{
  \frametitle{QGSP\_BERT W and W/O EMV and optical physics}
  \vskip -0.3cm
  \begin{figure}
    \includegraphics[width=0.85\columnwidth]<1>{g4_emv_opt_pip.eps}
    \includegraphics[width=0.85\columnwidth]<2>{g4_emv_opt_pim.eps}
  \end{figure}
  \vskip -0.2cm
  \begin{itemize}
    \scriptsize
  \item No visible difference whether EMV parameters are tuned or not ($\implies$ Use EMV)
  \item No visible difference whether optical physics is turned on or not ($\implies$ Do not use optical)
  \end{itemize}
  
}




\frame{
  \frametitle{}
  
  \begin{block}{}
    \begin{center}
      {\huge
        High precision neutron data \\
        }
      for neutron transport below 20~MeV
    \end{center}
  \end{block}

}

\frame{
  \frametitle{QGSP\_BERT\_EMV vs. QGSP\_BERT\_HP}
  \vskip -0.3cm
  \begin{figure}
    \includegraphics[width=0.85\columnwidth]<1>{g4_hp_opt_pip.eps}
    \includegraphics[width=0.85\columnwidth]<2>{g4_hp_opt_pim.eps}
  \end{figure}
  \vskip -0.2cm
  \begin{itemize}
    \scriptsize
  \item No visible difference whether HP neutron data is used ($\implies$ Do not Use HP)
  \end{itemize}
  
}








\frame{
  \frametitle{}
  \begin{block}{}
    \begin{center}
      {\huge
        Geant3 vs. Geant4 
        }
    \end{center}
  \end{block}
}

\frame{
  \frametitle{G4 \only<1>{$\pi^+$}\only<2>{$\pi^{-}$} G3 vs. QGSP\_BERT\_EMV vs. QGSP\_BIC\_EMV}
  \begin{figure}
    \includegraphics[width=0.85\columnwidth]<1>{g3vsg4_pip.eps}
    \includegraphics[width=0.85\columnwidth]<2>{g3vsg4_pim.eps}
  \end{figure}
}



\frame{
  \frametitle{}
  \begin{block}{}
    \begin{center}
      {\huge
        Backup
        }
    \end{center}
  \end{block}
}

\frame{
  \frametitle{G3 options}

  HADR=1: GEANH
  
  HADR=3: GEANH+NUCRIN (Default)
  \begin{itemize}
    \scriptsize
    \item GEANH: Transport
    \item NUCRIN: ?
  \end{itemize}
  

  HADR=4: FLUKA
  \begin{itemize}
    \scriptsize
    \item GEANH: Transport ?
    \item FLUKA: The FLUKA fragmentation model is utilized for interactions above the HETC limit
  \end{itemize}
  
  HADR=5: MICAP
  \begin{itemize}
    \scriptsize
    \item GEANH: Transport ?
    \item MICAP: Neutron code from the Monte-carlo-Ionization-Chamber-Analysis-Program for neutrons with a kinetic energy below 20 MeV.
  \end{itemize}
  
  
  HADR=6: GCALOR
  \begin{itemize}
    \scriptsize
    \item HETC: The High-Energy-Transport-Code is transporting charged hadrons up to an energy of 10 GeV through the materials of the setup
    \item FLUKA and MICAP are also used in their validity domain
  \end{itemize}

}

\frame{
  \frametitle{$\pi^+$ vs $\pi^-$ G4 QGSP, Bertini cascade, Full EM Physics + Optical physics (QGSP\_BERT\_OPTICAL) }
  \vskip -0.3cm
  \begin{figure}
    \includegraphics[width=0.85\columnwidth]{pvsm_QGSP_BERT_OPTICAL.eps}
  \end{figure}
}

\frame{
  \frametitle{$\pi^+$ vs $\pi^-$ G4 QGSP, Binary cascade, param. EM processes + Optical physics (QGSP\_BIN\_EMV\_OPTICAL) }
  \vskip -0.3cm
  \begin{figure}
    \includegraphics[width=0.85\columnwidth]{pvsm_QGSP_BERT_EMV_OPTICAL.eps}
  \end{figure}
}

\frame{
  \frametitle{$\pi^+$ vs $\pi^-$ G4 QGSP, Binary cascade, High Precision Neutron Package (QGSP\_BERT\_HP) }
  \vskip -0.3cm
  \begin{figure}
    \includegraphics[width=0.85\columnwidth]{pvsm_QGSP_BERT_HP.eps}
  \end{figure}
}

\end{document}

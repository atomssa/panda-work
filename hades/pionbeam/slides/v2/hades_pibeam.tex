\documentclass[20]{article}
\usepackage[latin1]{inputenc}
\usepackage[dvips]{graphicx,epsfig,color}
\usepackage{wrapfig,rotating}
\usepackage{amssymb,amsmath,array}

%\pagestyle{empty}

\begin{document}
%\title{Addition start detector }
%\author{Ermias Atomssa}

%\maketitle

Described below is a simulation study to determine how much
improvement one can gain in terms of reconstructed momentum and
position resolution by adding hit position information from the start
detector. The simulation is based on the pion beam line simulation
code HBeam.h (part of HYDRA?) that generates pions with a given
initial beam profile (object spot size, angular spread, momentum and
momentum spread ... ). The way the code is implemented, horizontal and
vertical initial beam spot sizes are the same. The code was modified
to implement the solution described in the pion beam TDR to
reconstruct, for each particle, its full state based on the crossing
positions in the transverse plane at the location of the two pion
tracker silicon detectors. In the simulation, they are located at 17~m
and 5.4~m from the HADES target.

In addition, a rudimentary digitization was introduced to the
transverse crossing positions on the silicon trackers in order to
account for the finite resolution of these detectors. The digitized
position of a track that passes through a given stripe is calculated
as the position of mid point of the stripe through which it
passes. These detectors are 10x10~cm$^2$ and have 128 strips in x and
y directions etched on either face, so the size of a strip comes to
0.78~mm. It is assumed that the reference trajectory passes through
the middle of the detector between stripes 63 and 64 (zero base), and
not in the middle of stripe 64. Because of spikes observed in solution
distributions with this scheme of digitization, the digitized position
is re-smeared again with a uniform random number generated with width
equal to the size of a strip.

\section{TDR method}
In order to validate the reconstruction algorithm and to demonstrate
the degradation of resolution due to various changes, the following
parameters were changed step by step: (a) Initial object size and
angular spread (b) digitization of pion tracker hit as described above
(c) addition of multiple scattering on the two pion trackers. The
following were the four steps that were simulated:

\begin{itemize}
  \item Step 0: Zero object spot size with zero angular spread, and
    non digitized position measurement, no multiple scattering
  \item Step 1: Nominal object spot size and angular spread
    (x,y$\rightarrow$0.5mm, $\theta\rightarrow$10~mrad, $\phi\rightarrow$50~mrad), with no
    digitization and multiple scattering
  \item Step 2: Nominal object spot size and angular spread, with
    digitized hits but no multiple scattering
  \item Step 3: Nominal object spot size and angular spread, with
    digitized hits and multiple scattering
\end{itemize}

Each simulation consisted of 10k events at reference momentum
$p_{ref}$=1.3~GeV/c. No acceptance cuts were applied. The momentum
offset was chosen uniformly among all integral values between -6\% and
6\% so each step had 13 distinct values of momentum offset
$\delta$. In a first step, the quantities that were tested were (a)
momentum resolution (b) position resolution at the HADES target and
(c) position resolution at the start detector. The main objective of
the study is to check if the inclusion of the hits in the start
detector are worth adding to the reconstruction procedure. If the
position resolution at the position of start detector is already
better than the resolution it offers ($\approx 14mm/sqrt{12} = 4mm$),
then there is no point in adding the position information from this
detector to the fit.

The first thing that was checked to make sure that the solution was
implemented correctly was to check that momenta and positions are
reconstructed perfectly in step 0. For this purpose, the reconstructed
momenta and x and y positions at the detector and HADES target were
plotted. These are shown in Figure~\ref{step0:pres} and
Figure~\ref{step0:xres} respectively which show that reconstruction
retrieves truth information perfectly. In Figure~\ref{step0:pres},
each color is from simulated pions at a given value of momentum
offset. These two figures indicate that the reconstruction algorithm
was implemented correctly.

\begin{figure}[tbp]
  \includegraphics[width=\columnwidth]{tc_pres_TDR_case0.eps}
  \caption{Reconstructed momentum distribution in Step 0. All momentum
    offsets represented by different colors are perfectly
    reconstructed.}
  \label {step0:pres}
\end{figure}


\begin{figure}[tbp]
  \includegraphics[width=\columnwidth]{tc_xres_dia_TDR_case0.eps}
  \caption{Reconstructed x position (horizontal) resolution
    distribution in Step 0 at the start detector. The difference
    between the reconstructed x position and the ``true'' x position
    is plotted for 12 (of 13) different momentum offset values. The
    same is found for x position both at the HADES target and y
    position at both the HADES target and start detector. The position
    reconstruction thus works for all momentum offsets.}
  \label {step0:xres}
\end{figure}

The next step was to switch to typical initial beam spot (object) size
and angular spread. The solution in the TDR assumes that initial
position of beam particles in the horizontal dimension is
negligible. Introducing the spread to this variable (and at the same
time spread in vertical object size and angular spreads) thus should
result in a finite resolution in reconstructed quantities as shown in
Figure~\ref{step1:pres} for reconstructed momentum and
Figure~\ref{step1:xres} for reconstructed horizontal position
resolution at the start detector.
                                   
\begin{figure}[tbp]
  \includegraphics[width=\columnwidth]{tc_pres_TDR_case1.eps}
  \caption{Reconstructed momentum distribution in Step 1. The effect
    of introducing finite object extension and angular spread is
    shown. }
  \label {step1:pres}
\end{figure}


\begin{figure}[tbp]
  \includegraphics[width=\columnwidth]{tc_xres_dia_TDR_case1.eps}
  \caption{Reconstructed x position (horizontal) resolution
    distribution in Step 1 at the start detector. The difference
    between the reconstructed x position and the ``true'' x position
    is plotted for 12 (of 13) different momentum offset values. The
    width is no longer zero, and reflects uncertainties introduced by
    the assumption of zero object extension in the horizontal
    direction.}
  \label {step1:xres}
\end{figure}

Notice that the position resolution distributions are not really
Gaussian and the fits by a Gaussian do not work in most cases. However
the width of the Gaussian fits are used as a rough estimate of the
reconstruction resolution for comparison. In Figure~\ref{step1:res},
the above mentioned figures are summarized by plotting the momentum
resolution (left panel) and position resolution (right panel) as a
function of momentum offset. In the position resolution plot, both
horizontal and vertical resolutions at the start detector and HADES
target locations are shown. At this point the vertical resolution is
negligible at its minimum, whereas the horizontal resolution has
sizable value (more than half size of a strip) even at its
minimum. However, if the conditions of this step were true, there
would be no need to include the start detector positions into a global
fit to reconstruct momentum.

\begin{figure}[tbp]
  \includegraphics[width=\columnwidth]{tc_res_TDR_case1.eps}
  \caption{Summary of momentum and position resolutions as a function
    of momentum offset in step 1.}
  \label {step1:res}
\end{figure}

In Step 2, the finite resolution of pion trackers was introduced. The
result is shown in Figure~\ref{step2:pres} through
Figure~\ref{step2:res}. The momentum resolution at its minimum worsens
from $\approx$0.5\% to $\approx$0.7\% as shown in
Figure~\ref{step2:res}. The position resolution is barely affected by
adding finite strip size of the pion trackers.


\begin{figure}[tbp]
  \includegraphics[width=\columnwidth]{tc_pres_TDR_case2.eps}
  \caption{Reconstructed momentum distribution in Step 2. The effect
    of introducing finite position measurement resolution at the
    silicon detector is shown. }
  \label {step2:pres}
\end{figure}

\begin{figure}[tbp]
  \includegraphics[width=\columnwidth]{tc_xres_dia_TDR_case2.eps}
  \caption{Reconstructed x position (horizontal) resolution
    distribution in Step 2 at the start detector. The difference
    between the reconstructed x position and the ``true'' x position
    is plotted for 12 (of 13) different momentum offset values. The
    width is no longer zero, and reflects uncertainties introduced by
    the assumption of zero object extension in the horizontal
    direction.}
  \label {step2:xres}
\end{figure}


\begin{figure}[tbp]
  \includegraphics[width=\columnwidth]{tc_res_TDR_case2.eps}
  \caption{Summary of momentum and position resolutions as a function
    of momentum offset for Step 2}
  \label {step2:res}
\end{figure}

In the last step (Step 3), multiple scattering emulation is enabled in
the two pion tracker detectors. The results shown in
Figure~\ref{step3:pres}, Figure~\ref{step3:xres} and
Figure~\ref{step3:res} exhibit a significant amount of position
resolution degradation, while the momentum resolution worsens further
by a small amount. The horizontal resolution is now of the order of
4mm at the minimum, about the same size as the resolution of start
detectors. Based on this one can expect that the inclusion of hit
position information from the start detector will only bring a
marginal improvement. 

\begin{figure}[tbp]
  \includegraphics[width=\columnwidth]{tc_pres_TDR_case3.eps}
  \caption{Reconstructed momentum distribution in Step 3. The effect
    of introducing multiple scattering in the silicon detectors is
    shown. }
  \label {step3:pres}
\end{figure}

\begin{figure}[tbp]
  \includegraphics[width=\columnwidth]{tc_xres_dia_TDR_case3.eps}
  \caption{Reconstructed x position (horizontal) resolution
    distribution in Step 3 at the start detector. The difference
    between the reconstructed x position and the ``true'' x position
    is plotted for 12 (of 13) different momentum offset values. The
    Gaussian fits are no longer working properly.}
  \label {step3:xres}
\end{figure}

\begin{figure}[tbp]
  \includegraphics[width=\columnwidth]{tc_res_TDR_case3.eps}
  \caption{Summary of momentum and position resolutions as a function
    of momentum offset for Step 3}
  \label {step3:res}
\end{figure}

\section{MINUIT fit on pion tracker and start detector position information}

The minuit fit is implemented to minimize the following $\chi^2$ function of initial state vector $V_0=(x_0,\theta_0,y_0,\phi_0,\delta)$

\begin{equation}
  \chi^2(V_0) = \sum^{3}_{d=1} \left( \frac{x^{d}_{M} - x^{d}_{T}(V_0, T^d)}{\delta x^{d}_{M}} \right)^2 + \sum^{3}_{d=1} \left( \frac{y^{d}_{M} - y^{d}_{T}(V_0, T^d)}{\delta y^{d}_{M}} \right)^2 
\end{equation}

where the index d runs over the three detector subsystems (two pion
trackers and the diamond start detector), and $(x^{d}_{M}, y^{d}_{M})$
is the horizontal and vertical position measurements on detector $d$
with errors $(\delta x^{d}_{M}, \delta y^{d}_{M})$. The errors are
taken as the size of the segmentation for the particular detector. The
expected positions $(x^{d}_{T}, y^{d}_{T})$ depend on the initial
state vector $V_0$, which are the variables the minimization is
performed on, and the transport coefficients $T^{d}_{ij}$ and
$T^{d}_{ijk}$ at the detector d that describe the spectrometric line
and are just constant parameters in the minimization. $x_{M}$ and
$y_{M}$ are given by (keeping only the most significant first and
second order terms of the transport equations)

\begin{equation}
  % Tx[0]->T11, Tx[1]->T12, Tx[2]->T14, Tx[3]->T16, Tx[4]->T116, Tx[5]->T126, Tx[6]->T146, Tx[7]->T166
  % par[0]->x, par[1]->theta, par[2]->y, par[3]->phi, par[4]->del 
  x^{d}_{T}(V_0, T^{d}) = T^{d}_{11} x_0 + T^{d}_{12}  \theta_0 + T^{d}_{14}  \phi_0 + T^{d}_{16}  \delta + T^{d}_{116}  x_0\delta + T^{d}_{126}  x_{0}\delta + T^{d}_{146}  \phi_0\delta + T^{d}_{166} \delta^2
\end{equation}

and 

\begin{equation}
  % Ty[0]->T32, Ty[1]->T33, Ty[2]->T34, Ty[3]->T36, Ty[4]->T336, Ty[5]->T346, Ty[6]->T366
  % par[0]->x, par[1]->theta, par[2]->y, par[3]->phi, par[4]->del   
  y^{d}_{T}(V_0, T^{d}) = T^{d}_{32}x_{0} + T^{d}_{33}y_{0} + T^{d}_{34}\phi_{0} + T^d_{36}\delta + T^d_{336}y_0\delta + T^d_{346}\phi_0\delta + T^d_{366}\delta^2;
\end{equation}


Once the initial state is known, the \emph{reconstructed} positions
at the start detector and the HADES target are calculated by using the
full second order transport equation and the respective theoretical
transport coefficients. These reconstructed values are compared to the
\emph{true} hit position.

The momentum resolution results are shown below for some
conditions.

Figure~\ref{step2bis:res} shows the momentum and position resolution
comparison using the TDR method and the MINUIT fitting method for step
2 described above: nominal initial beam spot size and angular spread,
with segmented pion trackers, and non segmented start detector. This
assumes perfect start detector with no segmentation so is not
realistic, but is shown as a reference. In this idealized case, the
MINIUT method performs significantly better, especially on momentum
resolution at large values of $|\delta|$.

\begin{figure}[tbp]
  \includegraphics[width=\columnwidth]{tc_res_case2.eps}
  \caption{Summary of momentum and position resolutions as a function
    of momentum offset for Step 2, comparing the results using TDR
    method (that doesn't use start detector info) and global MINUIT
    fit (uses perfect start detector information). No multiple scattering. }
  \label {step2bis:res}
\end{figure}

When finite resolution (segmentation) is added to the start detector,
the situation changes dramatically as shown in
Figure~\ref{step2a:res}. The segmentation size for the start detector
here is taken to be 14~mm, close to the real ``pixel'' size. All the
momentum resolution improvement is lost and the two methods give
comparable resolution, while the position resolution in the horizontal
direction worsens significantly compared to TDR method. Note however
that multiple scattering is not included in these comparisons.

\begin{figure}[tbp]
  \includegraphics[width=\columnwidth]{tc_res_case2a.eps}
  \caption{Summary of momentum and position resolutions as a function
    of momentum offset for Step 2, comparing the results using TDR
    method (that doesn't use start detector info) and global MINUIT
    fit (uses segmented start detector information). No multiple scattering. } 
  \label {step2a:res}
\end{figure}


Figure~\ref{step3a:res} shows what the TDR-MINUIT comparison looks
like when realistic detector segmentation is used and multiple
scattering is enabled in the two start detectors. The observation here
is that the the TDR solution position resolution in the horizontal
direction worsens while the MINIUT solution resolution is more or less
the same. Even so, this does not justify using MINUIT solution because
at its best it performs only as good as the TDR solution.

\begin{figure}[tbp]
  \includegraphics[width=\columnwidth]{tc_res_case3a.eps}
  \caption{Summary of momentum and position resolutions as a function
    of momentum offset for Step 3, comparing the results using TDR
    method (that doesn't use start detector info) and global MINUIT
    fit (uses segmented start detector information). Multiple
    scattering is enabled in the two pion tracker detectors. }
  \label {step3a:res}
\end{figure}

As a reference, Figure~\ref{step3:variations} shows what the
resolution could be if the segmentation of the start detector was
finer. Namely, it compares the MINUIT solution resolution for finer
segmentation's. One notices that for the vertical position resolution
and momentum resolution, the segmentation of the pion tracker doesn't
matter much. For the horizontal position resolution, one needs a
segmentation diamond detector that is four times finer to have an
appreciable advantage of using the global MINUIT fit.

\begin{figure}[tbp]
  \includegraphics[width=\columnwidth]{tc_res_var.eps}
  \caption{Summary of momentum and position resolutions as a function
    of momentum offset for Step 3, comparing the results of the global
    MINUIT fit when the segmentation of the start detector is
    varied. All other simulation conditions are constant: Nominal
    beam spot, MS turned on the two pion trackers which use nominal
    segmentation.}
  \label {step3:variations}
\end{figure}

Finally the effect of enabling multiple scattering in all three
detectors is shown in Figure~\ref{step3c:res}. This configuration is
the one that is closest to reality. The result suggests that no
appreciable gain in resolution (momentum or position) should be
expected from the usage of the start detector position information.

\begin{figure}[tbp]
  \includegraphics[width=\columnwidth]{tc_res_case3c.eps}
  \caption{Summary of momentum and position resolutions as a function
    of momentum offset for the most realistic simulation
    condition. True segmentation and MS is used in all three
    detectors. The beam initial condition is set to nominal values. }
  \label {step3c:res}
\end{figure}


\end{document}

\documentclass{beamer}
%\usepackage{beamerthemeshadow}
\usetheme{Madrid}

%\usepackage[utf8]{inputenc}
%\usepackage{xspace}
%\usepackage[T1]{fontenc}


\usepackage{appendixnumberbeamer}
\usepackage{amsmath}
\usepackage{amssymb}
\usepackage{textpos}

\begin{document}

\newcommand{\pnd}{\=PANDA }
\newcommand{\streama}{{\color{blue} Stream A~}}
\newcommand{\streamb}{{\color{red} Stream B~}}
\newcommand{\dm}{\mathrm{d}m}
\newcommand{\dpt}{\mathrm{d}p_T}
\newcommand{\pt}{p_T}
\newcommand{\tab}{\hspace{0.25cm}}

\newcommand{\pip}{{\normalsize$\pi^+$~}}
\newcommand{\pim}{{\normalsize$\pi^-$~}}
\newcommand{\pipm}{{\normalsize$\pi^{+}\pi^{-}$~}}
\newcommand{\epm}{{\normalsize$e^+e^-$}}
\newcommand{\piz}{{\normalsize$\pi^0$~}}
\newcommand{\jpsi}{{\normalsize$J/\psi$~}}
\newcommand{\jpsipbarp}{{\normalsize$J/\psi\rightarrow p\bar{p}$~}}

\newcommand{\pipT}{{\LARGE$\pi^+$~}}
\newcommand{\pimT}{{\LARGE$\pi^-$~}}
\newcommand{\pipmT}{{\LARGE$\pi^{+}\pi^{-}$~}}
\newcommand{\epmT}{{\LARGE$e^+e^-$}}
\newcommand{\pizT}{{\LARGE$\pi^0$~}}
\newcommand{\jpsiT}{{\LARGE$J/\psi$~}}
\newcommand{\jpsipbarpT}{{\LARGE$J/\psi\rightarrow p\bar{p}$~}}


\newcommand{\pintda}{$\pi$-N~TDA~}
\newcommand{\pintdas}{$\pi$-N~TDAs~}
\newcommand{\piantda}{$\pi$-$\bar{N}$~TDA~}
\newcommand{\piantdas}{$\pi$-$\bar{N}$~TDAs~}

\newcommand{\pizjpsi}{{\normalsize$\pi^0J/\psi$~}}
\newcommand{\pizpipm}{{\normalsize$\pi^0$\pipm}}
\newcommand{\pizjpsiT}{{\LARGE$\pi^0J/\psi$~}}
\newcommand{\pizpipmT}{{\LARGE$\pi^0$\pipmT}}

\newcommand{\sigrxn}{{\normalsize$\bar{p}p\rightarrow\pi^0J/\psi\rightarrow\pi^0e^+e^-$~}}
\newcommand{\sigrxnepem}{{\normalsize$\bar{p}p\rightarrow\pi^0\gamma^*\rightarrow\pi^0e^+e^-$~}}
\newcommand{\bgrxn}{{\normalsize$\bar{p}p\rightarrow\pi^0\pi^{+}\pi^{-}$~}}
\newcommand{\sigrxnT}{{\LARGE$\bar{p}p\rightarrow\pi^0J/\psi\rightarrow\pi^0e^+e^-$~}}
\newcommand{\sigrxnepemT}{{\LARGE$\bar{p}p\rightarrow\pi^0\gamma^*\rightarrow\pi^0e^+e^-$~}}
\newcommand{\bgrxnT}{{\LARGE$\bar{p}p\rightarrow\pi^0\pi^{+}\pi^{-}$~}}


\newcommand{\bgcut}{$-0.5<t[$GeV$^2]<0.6$, $2.96<M_{inv}[$GeV/c$^2] < 3.22$~}
\newcommand{\tcut}{$-0.5<t[$GeV$^2]<0.6$~}
\newcommand{\ucut}{$-0.5<u[$GeV$^2]<0.6$~}

%\renewcommand{\inserttotalframenumber}{17}

\title[\pnd Group Meeting, TDA]{Full MC simulation of \sigrxn and \bgrxn}

\author[Ermias ATOMSSA]{\pnd Group Meeting\\ \vskip 0.5cm Ermias ATOMSSA}

\date[October 13, 2014]{October 13, 2014}

\institute[IPNO]{Institut de Physique Nucl\'eaire d'Orsay}

\frame{\titlepage}

\frame{
  \frametitle{Suggestion from last week}
  %\begin{block}{}
    \begin{itemize}
      \scriptsize
    \item $\pi^0$ selection: Instead of nearest to $m^{PDG}_{\pi^0}$, choose most back to back
    \item Check for improvements in kinematic fits in doing so
    \item Any difference in opening angle distribution between signal and background
    \item Procedure:
      \begin{itemize}
        \scriptsize
      \item Select a charged pair ($e^{+}e^{-}$ or $\pi^{+}\pi^{-}$) in the event within $J/\psi$ mass window (2.96-3.22~GeV/c)
      \item For each $\gamma-\gamma$ pair that can be formed, look at
        \begin{itemize}
          \tiny
        \item opening angle between the $\gamma-\gamma$ pair and the charged track pair in the CM frame
        \item Total invariant mass ($M^{inv}_{(\gamma-\gamma)-(e^{+}e^{-})}$ or $M^{inv}_{(\gamma-\gamma)-(\pi^{+}\pi^{-})}$)
        \end{itemize}
      \item Select either the most back to back pair or the pair with $M_{inv}$ closest to $\sqrt{s}=3.5~GeV/c^2$
      \item Its not possible to apply both conditions at the same
        time, though it might be possible to do some type of distance
        calculation or devise a cut in the $OA-M_{inv}$ plane
      \item After selecting the best $\pi^0$ check if there is any change in the kinmeatical fit results
      \end{itemize}
    \end{itemize}
 %\end{block}
}


\frame{
  \frametitle{}
  \begin{figure}
    \includegraphics[width=\columnwidth]{fig_p1.pdf}
  \end{figure}
}

\frame{
  \frametitle{}
  \begin{figure}
    \includegraphics[width=\columnwidth]{fig_p2.pdf}
  \end{figure}
}

\frame{
  \frametitle{}
  \begin{figure}
    \includegraphics[width=\columnwidth]{fig_p3.pdf}
  \end{figure}
}

\frame{
  \frametitle{}
  \begin{figure}
    \includegraphics[width=\columnwidth]{fig_p4.pdf}
  \end{figure}
}

\frame{
  \frametitle{}
  \begin{figure}
    \includegraphics[width=\columnwidth]{fig_p5.pdf}
  \end{figure}
}

\frame{
  \frametitle{}
  \begin{figure}
    \includegraphics[width=\columnwidth]{fig_p6.pdf}
  \end{figure}
}

\frame{
  \frametitle{}
  \begin{figure}
    \includegraphics[width=\columnwidth]{fig_p7.pdf}
  \end{figure}
}

\frame{
  \frametitle{Conclusion}
  \begin{block}{}
    \begin{itemize}
      \scriptsize
    \item After the mass cut it is very difficult to distinguish between the $e^{+}e^{-}\pi^0$ and $\pi^{+}\pi^{-}\pi^0$ events
    \end{itemize}
  \end{block}
  \begin{block}{Things to do}
    \begin{itemize}
      \scriptsize
    %\item $\pi^0$ selection: Instead of nearest to $m^{PDG}_{\pi^0}$, choose most back to back
    \item $\pi^{\pm}$ efficiency: How will it affect kinematic distributions?
    \item Other background sources: What other kinds of events does DPM generate
    \item Kinematic fit: quantify the efficiency/rejection of any cut
    \item Other buggy behaviors: No truth match for $\pi^0$s from BG
    \end{itemize}
  \end{block}
}

\appendix

\frame{
  \frametitle{}
  \begin{figure}
    \includegraphics[width=\columnwidth]{fig_p8.pdf}
  \end{figure}
}

\frame{
  \frametitle{}
  \begin{figure}
    \includegraphics[width=\columnwidth]{fig_p9.pdf}
  \end{figure}
}





\end{document}

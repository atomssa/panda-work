\documentclass{beamer}
%\usepackage{beamerthemeshadow}
\usetheme{Madrid}

%\usepackage[utf8]{inputenc}
%\usepackage{xspace}
%\usepackage[T1]{fontenc}


\usepackage{appendixnumberbeamer}
\usepackage{amsmath}
\usepackage{amssymb}
\usepackage{textpos}

\begin{document}

\newcommand{\pnd}{\=PANDA }
\newcommand{\streama}{{\color{blue} Stream A~}}
\newcommand{\streamb}{{\color{red} Stream B~}}
\newcommand{\dm}{\mathrm{d}m}
\newcommand{\dpt}{\mathrm{d}p_T}
\newcommand{\pt}{p_T}
\newcommand{\tab}{\hspace{0.25cm}}

\newcommand{\pip}{{\normalsize$\pi^+$~}}
\newcommand{\pim}{{\normalsize$\pi^-$~}}
\newcommand{\pipm}{{\normalsize$\pi^{+}\pi^{-}$~}}
\newcommand{\epm}{{\normalsize$e^+e^-$}}
\newcommand{\piz}{{\normalsize$\pi^0$~}}
\newcommand{\jpsi}{{\normalsize$J/\psi$~}}
\newcommand{\jpsipbarp}{{\normalsize$J/\psi\rightarrow p\bar{p}$~}}

\newcommand{\pipT}{{\LARGE$\pi^+$~}}
\newcommand{\pimT}{{\LARGE$\pi^-$~}}
\newcommand{\pipmT}{{\LARGE$\pi^{+}\pi^{-}$~}}
\newcommand{\epmT}{{\LARGE$e^+e^-$}}
\newcommand{\pizT}{{\LARGE$\pi^0$~}}
\newcommand{\jpsiT}{{\LARGE$J/\psi$~}}
\newcommand{\jpsipbarpT}{{\LARGE$J/\psi\rightarrow p\bar{p}$~}}


\newcommand{\pintda}{$\pi$-N~TDA~}
\newcommand{\pintdas}{$\pi$-N~TDAs~}
\newcommand{\piantda}{$\pi$-$\bar{N}$~TDA~}
\newcommand{\piantdas}{$\pi$-$\bar{N}$~TDAs~}

\newcommand{\pizjpsi}{{\normalsize$\pi^0J/\psi$~}}
\newcommand{\pizpipm}{{\normalsize$\pi^0$\pipm}}
\newcommand{\pizjpsiT}{{\LARGE$\pi^0J/\psi$~}}
\newcommand{\pizpipmT}{{\LARGE$\pi^0$\pipmT}}

\newcommand{\sigrxn}{{\normalsize$\bar{p}p\rightarrow\pi^0J/\psi\rightarrow\pi^0e^+e^-$~}}
\newcommand{\sigrxnepem}{{\normalsize$\bar{p}p\rightarrow\pi^0\gamma^*\rightarrow\pi^0e^+e^-$~}}
\newcommand{\bgrxn}{{\normalsize$\bar{p}p\rightarrow\pi^0\pi^{+}\pi^{-}$~}}
\newcommand{\sigrxnT}{{\LARGE$\bar{p}p\rightarrow\pi^0J/\psi\rightarrow\pi^0e^+e^-$~}}
\newcommand{\sigrxnepemT}{{\LARGE$\bar{p}p\rightarrow\pi^0\gamma^*\rightarrow\pi^0e^+e^-$~}}
\newcommand{\bgrxnT}{{\LARGE$\bar{p}p\rightarrow\pi^0\pi^{+}\pi^{-}$~}}

\newcommand{\bgcut}{$-0.5<t[$GeV$^2]<0.6$, $2.96<M_{inv}[$GeV/c$^2] < 3.22$~}
\newcommand{\tcut}{$-0.5<t[$GeV$^2]<0.6$~}
\newcommand{\ucut}{$-0.5<u[$GeV$^2]<0.6$~}

%\renewcommand{\inserttotalframenumber}{17}

\title[\pnd Group Meeting, TDA]{Full MC simulation of \sigrxn and \bgrxn}

\author[Ermias ATOMSSA]{\pnd Group Meeting\\ \vskip 0.5cm Ermias ATOMSSA}

\date[October 13, 2014]{October 13, 2014}

\institute[IPNO]{Institut de Physique Nucl\'eaire d'Orsay}

\frame{\titlepage}

\frame{
  \frametitle{Main comments from last week}
  %\begin{block}{}
    \begin{itemize}
      \scriptsize
    \item Apply cuts on $\gamma$ (E and OA)before making pairs
    \item Use MC truth to guide the cuts
    \item For kinematic cut, use $\Delta\theta$ instad of opening angle
    \item Based on the comment the flow is now
      \begin{itemize}
        \scriptsize
      \item Select $\pi^0$ candidates
        \begin{itemize}
          \tiny
        \item Decide E and OA cuts by comparing distributions of all candidates to those that have full MC match
        \item MC matching is at pair level (the $\gamma$ pair is checked to have come from the primary $\pi^0$)
        \item The count of pairs that has a primary match is used as normalization for efficiency numbers
        \item Finally a mass cut is also applied to narrow down the selection further
        \end{itemize}

      \item Select $J/\psi$ candidates
        \begin{itemize}
          \tiny
        \item Only mass cut is applied for the $J/\psi$ because combinatorics is not as bad as for $\pi^0$
        \item Full MC matching is at pair level for normalization
        \end{itemize}
      \item At this point a selection of $\pi^0$ and $J/\psi$ candidates is available
      \item All $\pi^0-J/\psi$ candidate combinations are considered
      \item $\Delta\theta$ and total mass of $\pi^0$ and J/$\psi$ is tested for further selection
      \end{itemize}
    \item Work on proper kinematic fits is still underway
    \item MC matching was implemented (the default matching is poor - namely any $\pi^0$ can match even if it is not primary)
    \item Candidate reason why matching doesn't work with background: too many neutral candidates. The matching algoritm never gets to the photons that decayed from primary $\pi^0$
    \end{itemize}
 %\end{block}
}


\frame{
  \frametitle{}
  \begin{figure}
    \includegraphics[width=\columnwidth]{fig_p1.pdf}
  \end{figure}
}

\frame{
  \frametitle{}
  \begin{figure}
    \includegraphics[width=\columnwidth]{fig_p2_file1_cut0.pdf}
  \end{figure}
}

\frame{
  \frametitle{}
  \begin{figure}
    \includegraphics[width=\columnwidth]{fig_p2_file1_cut8.pdf}
  \end{figure}
}

\frame{
  \frametitle{Efficiencies}
  \begin{center}
  \begin{tabular}{|c|c|c|}
%    \begin{table}
    \hline
      E cut & w/o Mcut & w/Mcut \\
      \hline
      E$>$ 0.025 & 90.11 &75.24 \\
      E$>$ 0.05 &  87.67 &74.84 \\
      E$>$ 0.075 & 85.89 &74.1 \\
      E$>$ 0.1 &   84.32 &73.27 \\
      E$>$ 0.005 & 82.97 &72.4 \\
      E$>$ 0.01 &  76.83 &68.49 \\
      E$>$ 0.015 & 71.37 &64.51 \\
      E$>$ 0.02 &  66.39 &60.77 \\
      \hline
%    \end{table}
  \end{tabular}
  \end{center}
}
\frame{
  \frametitle{}
  \begin{figure}
    \includegraphics[width=\columnwidth]{fig_p3.pdf}
  \end{figure}
}

\frame{
  \frametitle{}
  \begin{figure}
    \includegraphics[width=\columnwidth]{fig_p4.pdf}
  \end{figure}
}

\frame{
  \frametitle{}
  \begin{figure}
    \includegraphics[width=\columnwidth]{fig_p5_sub0.pdf}
  \end{figure}
}

\frame{
  \frametitle{}
  \begin{figure}
    \includegraphics[width=\columnwidth]{fig_p5_sub1.pdf}
  \end{figure}
}



%\frame{
%  \frametitle{}
%  \begin{figure}
%    \includegraphics[width=\columnwidth]{fig_p6.pdf}
%  \end{figure}
%}
%
%\frame{
%  \frametitle{}
%  \begin{figure}
%    \includegraphics[width=\columnwidth]{fig_p7.pdf}
%  \end{figure}
%}
%
%\frame{
%  \frametitle{Conclusion}
%  \begin{block}{}
%    \begin{itemize}
%      \scriptsize
%    \item After the mass cut it is very difficult to distinguish between the $e^{+}e^{-}\pi^0$ and $\pi^{+}\pi^{-}\pi^0$ events
%    \end{itemize}
%  \end{block}
%  \begin{block}{Things to do}
%    \begin{itemize}
%      \scriptsize
%    %\item $\pi^0$ selection: Instead of nearest to $m^{PDG}_{\pi^0}$, choose most back to back
%    \item $\pi^{\pm}$ efficiency: How will it affect kinematic distributions?
%    \item Other background sources: What other kinds of events does DPM generate
%    \item Kinematic fit: quantify the efficiency/rejection of any cut
%    \item Other buggy behaviors: No truth match for $\pi^0$s from BG
%    \end{itemize}
%  \end{block}
%}

\appendix

%\frame{
%  \frametitle{}
%  \begin{figure}
%    \includegraphics[width=\columnwidth]{fig_p8.pdf}
%  \end{figure}
%}
%
%\frame{
%  \frametitle{}
%  \begin{figure}
%    \includegraphics[width=\columnwidth]{fig_p9.pdf}
%  \end{figure}
%}





\end{document}
